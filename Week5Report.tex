\section{Week 5 (18) Activity Report W/e 05.05.19}
Travelled to Scotland on Sunday 05.05.19.
\subsection{Progress review meeting 1}
Meeting with VW and AMW Thursday 02.05.18. The salient points were:
\begin{itemize}
    \item Follow the MaNGA DAP tutorial.
    \item Report results section could include:
    \begin{itemize}
        \item Images (scaled in kpc)
        \item Kinematic PA histograms
        \item K-S test analysis: rPSB-cPSB; rPSB-control; cPSB-control
        \item Radon transform analysis: process velocity field maps following \citet{2018MNRAS.480.2217S}.
        \item Smoothed velocity fields, following \citet{2008ApJ...682..231S} ? (this is more advanced than the Radon transform project).
    \end{itemize}
    \item Use ZDIST, RE in calculations. 
    \item Perform cosmological calculations for angular-diameter distances etc.
    \item Compare RE of cPSBs-rPSBs.
    \item Use 11 weeks on M.Sc. project. Research paper work after that.
    \item Follow-up meeting Wednesday next week.
    \item Is there anything significant that I have missed?
\end{itemize}

\subsection{Computing \& Data Analysis}
\begin{itemize}
    \item Received control sample FITS files removing cross-matched PSBs and Controls and vice-versa. TODO: check to confirm that PSB lists have been cleaned up.    
    \item Looked at the Radon code on GitHub. Radon requires IDL. TODO: Access IDL on Astrolab machines following IT's instructions.
    \item {Plotted normalised histograms of S\'ersic index, stellar mass and redshift for cPSBs and rPSBs. Submitted notebook.}
    \item Used \texttt{Astropy.cosmology} package to calculate angular-diameter distances and arcsecond to kpc scale conversions. Submitted notebook \texttt{AstroPy-units-demo} for review.
\end{itemize}

\subsection{Reading}
\begin{itemize}
    \item Revised version of Chen et al. (in preparation) - pub2.pdf.
    \item Further review of \citet{2018MNRAS.480.2217S}.
\end{itemize}