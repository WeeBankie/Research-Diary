%%% Research Diary - Entry
%%% Template by Mikhail Klassen, April 2013
%%% 
\documentclass[ceqn,usenatbib]{mnras}
%%% \documentclass[ceqn,usenatbib]

\newcommand{\workingDate}{\textsc{2013 $|$ January $|$ 01}}
\newcommand{\userName}{John Proctor}
\newcommand{\institution}{Your University}
\usepackage{researchdiary_png}
% To add your univeristy logo to the upper right, simply
% upload a file named "logo.png" using the files menu above.



\begin{document}
\univlogo

\title{Research Diary - Example Entry}

{\Huge January 20}\\[5mm]

\textit{N.B.: The following is a sample entry from Mikhail Klassen's research diary. It is intended to be illustrative of how WriteLaTeX can be used the keep track of your research progress. Some names have been removed from this document for privacy.}

\section*{Initial conditions for the turbulent molecular cloud run}

\subsection*{Inner radius}

The density profile follows an $r^{-3/2}$ power law. To avoid a singularity at the center, an interpolation is done over a radius. This inner radius is defined in the parameter file. It should follow the prescription of a singular isothermal sphere (see Binney \& Tremaine p.305), which is also the definition of the King radius:
\begin{equation}
r_0 \equiv \sqrt{\frac{9\sigma^2}{4\pi G\rho_0}}
\end{equation}
where $\sigma$ is the velocity dispersion and could be estimated as $\sigma = \mathcal{M} c_s$, where $c_s = \sqrt{\gamma P/\rho} = \sqrt{\gamma k_B T / \mu}$ is the sound speed.

The isothermal sound speed in our simulation was estimated
\begin{equation}
c_s = \sqrt{\frac{k_b T}{\mu m_p}}
\end{equation}
I'm unsure why a factor of $\gamma$ was not included. For 30 K, this gives a sound speed of about 34000 cm/s or 0.34 km/s. At a Mach number of 5, this gives a supersonic dispersion of $\sigma$ = 1.7 km/s

This gives an inner radius of $r_0 \approx$ 1.595e17. 

\subsection*{Rotation}

Set the same ratio of rotational to gravitational energy $\beta$ as in Peters et al. 2010a. According to Goodman et al. (1993), this is given by (see equation 6):
\begin{equation}
\beta = \frac{1}{4 \pi G \rho_0} \omega^2
\end{equation}
In practice we can probably use the central density $\rho_c$ instead of determining an average density $\rho_0$. Looking at the numbers from other simulations, we could use an $\omega$ of 1.3e-14.

The link to the Goodman et al. (1993) paper:\\
{\tt http://adsabs.harvard.edu/cgi-bin/bib\_query?1993ApJ...406..528G}

We want to complete our simulation with a similar $\beta$ to check if disks form in the turbulent environment.

The $\omega$ necessary to produce a $\beta = 0.05$ would be
\begin{equation}
\omega = \sqrt{4 \pi G \rho_0 \beta} \approx 7.15\times 10^{-13}
\end{equation}
using $\rho_0 = \rho_c = 1.22\times10^{-17}$.

After testing this, however, I found that the rotation was much too fast. Perhaps using $\rho_0 = \rho_c$ was not a very good assumption at all, since $\rho_c$ is orders of magnitude larger than the average. I wrote a little Python script that sums up all the mass inside the outer radius and divides it by the total volume, defined by the outer radius. In this case, for an outer radius of $5.97402\times 10^{18}$ cm and about 1000 $\Msun$, we get an average density of $2.96415\times 10^{-21}$ g/cm$^3$, which gives us $\omega = 1.114 \times 10^{-14}$.

\section{Week 1 Activity Report}
Week ending 31.03.19

\subsection{Summary}
Not a lot happening this week. Tried editing this file on GitHub. This has worked OK.

\subsection{MaNGA}
\begin{enumerate}
    \item MaNGA Python tutorials. Successfully completed the Python RSS and CUBE dataset examples.
    \item Visualised CUBE example in DS9
    \item Problem: Could not access Marvin in Windows (this is a known open issue on GitHub). Looks like I would need to go over to Linux to get around this.
\end{enumerate}

\subsection{Reading}
\begin{enumerate}
    \item Marvin documentation
    \item MaNGA DPR documentation
    \item Refresher on galaxy evolution and post-starburt galaxies \citep{sparke2007galaxies}
\end{enumerate}    




%%%%%%%%%%%%%%%%%%%% REFERENCES %%%%%%%%%%%%%%%%%%
% The best way to enter references is to use BibTeX:
\bibliographystyle{mnras}
%\bibliography{example} % if your bibtex file is called example.bib
\bibliography{JPbib2019} 



\end{document}